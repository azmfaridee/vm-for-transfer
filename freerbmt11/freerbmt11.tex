% File freebmt11.tex
% 17 September 2010
% Contact: freerbmt@dlsi.ua.es

\documentclass[11pt]{article}
\usepackage{freerbmt11}
\usepackage[utf8x]{inputenc}
\usepackage{times}
\usepackage{natbib}
\usepackage{url}
\usepackage{latexsym}

% \bibpunct{(}{)}{;}{A}{,}{,}
% \bibdata{mybib.bib}

\title{Speeding up Apertium: A Compiler and a Virtual Machine for the
  Transfer Module}

\author{Abu Zaher Md. Faridee\\
  Dept. of Comp. Science and Eng. \\
  Bangladesh Univ. of Eng. and Tech. \\
  Dhaka 1000, Bangladesh \\
  {\tt zaher14@gmail.com} \And
  Sergio Ortiz Rojas \\
  Dept. Llenguatges i Sist. Informàtics \\
  Universitat d'Alacant \\
  Alacant. E-03070 \\  
  {\tt sergio@prompsit.com}}

\date{}

\begin{document}

\maketitle

\begin{abstract}
  Apertium is a rule based, shallow transfer machine translation
  platform. While already a very robust architecture,  the XML based
  shallow transfer mechanism suffers from slowdown in case of complex
  transfer based language pairs. A new architechture is proposed to
  circumvent the issue. A compiler is proposed that will create
  pseudo-assembly files from the XML transfer files. A virtual machine
  will run instructions from this transfer file, speeding up the
  transfer system by taking advantage of jump instructions in the
  precalculated address space. Simple performace evaluation is also
  presented  along with future work directions.
\end{abstract}


\section{Introduction}
\label{sec:introduction}

% The following instructions are directed to authors of papers accepted
% for publication in the FREERBMT-11 proceedings.

% All authors are required to adhere to these specifications. Since the
% proceedings will appear in hardcopy and electronic form, authors are
% required to provide a Portable Document Format (PDF) version of their
% papers, {\bf with no page numbering}. The proceedings will be printed
% on A4 paper.

{\itshape{A brief intro about Apertium and its pipeline, some figure
    showing the whole system would be good enough}}

\section{The Transfer System}
\label{sec:transfer-system}

{\itshape{Mode detailed info about the transfer system}}

% \section{Using the provided style files}

% The easiest way to correctly format your paper is to use one of the
% style files provided on the workshop web page. For instance, \LaTeX{}
% users can use the {\bf freerbmt11.sty} style file.

% Figure \ref{latex-skeleton} shows a skeleton document for \LaTeX{}
% users.  To use it, simply copy the text in the figure to a file and
% then replace the parts flagged by {\bf TODO} as appropriate for your
% document.  Figure \ref{bibtex-skeleton} shows the simple Bib\TeX{}
% file used by the skeleton document.

% By using these style files most of the requirements discussed in
% the following sections will be automatically handled. Nonetheless,
% you should review the sections to become familiar with the
% requirements.

% \begin{figure}

%   ........................................................................
%   start of embedded latex example
%   \begin{small}
% \begin{verbatim}
% \documentclass[11pt]{article}
% \usepackage{freerbmt11}
% \usepackage[utf8x]{inputenc}
% \usepackage{natbib}
% \usepackage{times}
% \usepackage{url}
% \usepackage{latexsym}

% \title{TODO: put title here}

% \author{Author1 \\ address \\ {\tt email} 
% \And Author2 \\ address \\ {\tt email} \\
% TODO: fill in the author names here}

% \date{}

% \begin{document}
% \maketitle

% \begin{abstract}
%   TODO: put the abstract here
% \end{abstract}

% \section{Introduction}
% TODO: add introduction; and,
% perhaps cite FSM Book \citep{Beesley2003}

% \section{TODO: add body of paper}

% \section{Conclusion}
% TODO: add conclusion

% \section*{Acknowledgements}
% TODO: add acknowledgements

% \bibliographystyle{apalike}
% %% TODO: use base name of your .bib file
% \bibliography{my-citations}     

% \end{document}
% \end{verbatim}
%   \end{small}
%   %%   end of embedded latex example
%   %%   ........................................................................

%   \caption{Skeleton \LaTeX{} document illustrating use of {\bf
%   freerbmt11.sty} and {\bf natbib}.}
%   \label{latex-skeleton}
% \end{figure}


% \begin{figure}
%   \begin{small}
% \begin{verbatim}
% @Article{Miller-90,
% author =   {G. Miller},
% title =    {Special Issue on {WordNet}},
% journal =  {International Journal of
% Lexicography},
% year =     {1990},
% volume =   {3(4)},
% }
% \end{verbatim}
%   \end{small}
%   \caption{Bib\TeX{} file {\bf freerbmt11.bib} used in skeleton document}
%   \label{bibtex-skeleton}
% \end{figure}

% \section{General Instructions}

% Manuscripts must be in two-column format.  Exceptions to the
% two-column format include the title, authors' names and complete
% addresses, which must be centered at the top of the first page, and any
% full-width figures or tables (see the guidelines in
% Subsection~\ref{ssec:first}).  {\bf Type single-spaced.}  Use only one
% side of the page. Start all pages directly under the top margin.  See
% the guidelines later regarding formatting the first page.

% {\bf Do not print page numbers on the manuscript.}  

% The maximum length of a manuscript is eight ($8$) pages, printed
% single-sided (see Subsection~\ref{ssec:layout} for exact page layout
% and Section~\ref{sec:length} for additional information on the maximum
% number of pages).


% \section{Format of Electronic Manuscript}
% \label{sect:pdf}

% For the production of the electronic manuscript you must use Adobe's
% Portable Document Format (PDF). This format can be generated from
% postscript files: on Unix systems, you can use {\tt ps2pdf} for this
% purpose; under Microsoft Windows, Adobe's Distiller can be used.  Note
% that some word processing programs generate PDF which may not include
% all the necessary fonts (esp. tree diagrams, symbols). When you print
% or create the PDF file, there is usually an option in your printer
% setup to include none, all or just non-standard fonts.  Please make
% sure that you select the option of including ALL the fonts.  {\em
% Before sending it, test your {\/\em PDF} by printing it from a
% computer different from the one where it was created}. Moreover,
% some word processor may generate very large postscript/PDF files,
% where each page is rendered as an image. Such images may reproduce
% poorly.  In this case, try alternative ways to obtain the postscript
% and/or PDF.  One way on some systems is to install a driver for a
% postscript printer, send your document to the printer specifying
% ``Output to a file'', then convert the file to PDF.

% For reasons of uniformity, Adobe's {\bf Times Roman} font should be
% used. In \LaTeX2e{} this is accomplished by putting

% \begin{quote}
% \begin{verbatim}
% \usepackage{times}
% \usepackage{latexsym}
% \end{verbatim}
% \end{quote}
% in the preamble.

% Additionally, it is   important to specify  {\bf
% A4 format}  when formatting the paper.

% \subsection{Layout}
% \label{ssec:layout}

% Print the manuscript two columns to a page, in the manner these
% instructions are printed. 

% \subsection{The First Page}
% \label{ssec:first}

% Centre the title, author's name(s) and affiliation(s) across both
% columns. Do not use footnotes for affiliations.  Do not include the
% paper ID number that was assigned during the submission process.  Use
% the two-column format only when you begin the abstract.

% {\bf Title}: Place the title centered at the top of the first page, in
% a 15-point bold font. Long titles should be typed on two lines without
% a blank line intervening. Approximately, put the title at 1in from the
% top of the page, followed by a blank line, then the author's names(s),
% and the affiliation on the following line.  Do not use only initials
% for given names (middle initials are allowed). The affiliation should
% contain the author's complete address, and if possible an electronic
% mail address. Leave about 0.75in between the affiliation and the body
% of the first page.

% {\bf Abstract}: Type the abstract at the beginning of the first
% column.  The width of the abstract text should be smaller than the
% width of the columns for the text in the body of the paper by about
% 0.25in on each side.  Centre the word {\bf Abstract} in a 12 point
% bold font above the body of the abstract. The abstract should be a
% concise summary of the general thesis and conclusions of the paper.
% It should be no longer than 200 words.

% {\bf Text}: Begin typing the main body of the text immediately after
% the abstract, observing the two-column format as shown in the present
% document. Type single spaced.

% {\bf Indent} when starting a new paragraph. For reasons of uniformity,
% use Adobe's {\bf Times Roman} fonts, with 11 points for text and
% subsection headings, 12 points for section headings and 15 points for
% the title. If Times Roman is unavailable, use {\bf Computer Modern
% Roman} (\LaTeX2e{}'s default; see section \ref{sect:pdf} above).
% Note that the latter is about 10\% less dense than Adobe's Times Roman
% font.

% \subsection{Sections}

% {\bf Headings}: Type and label section and subsection headings in the
% style shown on the present document.  Use numbered sections (Arabic
% numerals) in order to facilitate cross references. Number subsections
% with the section number and the subsection number separated by a dot,
% in Arabic numerals. Do not number subsubsections.

% {\bf Citations}: Citations within the text should appear in
% parentheses as~\cite{Beesley2003} or, if the author's name appears in
% the text itself, as Gusfield~\shortcite{Gusfield97}.  Append lowercase
% letters to the year in cases of ambiguities.  Treat double authors as
% in~\cite{Aho72}, but write as in~\cite{Chandra81} when more than two
% authors are involved.  Collapse multiple citations as
% in~\cite{Gusfield97,Aho72}.

% \textbf{References}: Gather the full set of references together under
% the heading {\bf References}; place the section before any Appendices,
% unless they contain references. Arrange the references alphabetically
% by first author, rather than by order of occurrence in the text.
% Provide as complete a citation as possible, using a consistent format,
% such as the one for {\em Computational Linguistics\/} or the one in
% the {\em Publication Manual of the American Psychological
% Association\/}~\cite{APA83}.  Use of full names for authors rather
% than initials is preferred.  A list of abbreviations for common
% computer science journals can be found in the ACM {\em Computing
% Reviews\/}~\cite{ACM83}.

% The provided \LaTeX{} and Bib\TeX{} style files roughly fit the
% American Psychological Association format, allowing regular citations, 
% short citations and multiple citations as described above.

% {\bf Appendices}: Appendices, if any, directly follow the text and the
% references (but see above).  Letter them in sequence and provide an
% informative title: {\bf Appendix A. Title of Appendix}.

% The \textbf{Acknowledgements} section should go as the last section
% immediately before the references.  Do not number the acknowledgements
% section.

% \subsection{Footnotes}

% {\bf Footnotes}: Put footnotes at the bottom of the page. They may be
% numbered or referred to by asterisks or other symbols.\footnote{This
% is how a footnote should appear.} Footnotes should be separated from
% the text by a line.\footnote{Note the line separating the footnotes
% from the text.}

% \subsection{Graphics}

% {\bf Illustrations}: Place figures, tables, and photographs in the
% paper near where they are first discussed, rather than at the end, if
% possible.  Wide illustrations may run across both columns. Do not use
% colour illustrations as they may reproduce poorly.

% {\bf Captions}: Provide a caption for every illustration; number each
% one sequentially in the form: ``Figure 1. Caption of the Figure.''
% ``Table 1.  Caption of the Table.''  Type the captions of the figures
% and tables below the body, using 11 point text.


% \section{Length of Submission}
% \label{sec:length}

% Eight pages ($8$) is the maximum length of papers for the FREERBMT-11
% main workshop. All illustrations, references, and appendices must be
% accommodated within these page limits, observing the formatting
% instructions given in the present document.  Papers that do not
% conform to the specified length and formatting requirements will be
% returned to the author.

\section{Acknowledgements}

% This, and the other FREERBMT formatting documents, draws heavily on
% the instructions for authors provided by TMI which in turn draws from
% the EAMT for their EAMTCLAW-2003 conference in Dublin. Thanks for
% their understanding.

\bibliographystyle{apalike}
\bibliography{freerbmt11}

\end{document}
